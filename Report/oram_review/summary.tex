\section{总结与展望}
\subsection{ORAM小结}
\subsubsection{优劣分析}
表\ref{tab:adv}中,列出了几种ORAM模型的优缺点。
\begin{table}[H]
    \centering
    \begin{tabular}{|c|p{0.4\textwidth}|p{0.4\textwidth}|}
        \hline
        ORAM Type&优点&缺点\\
        \hline
        \multirow{3}*{Basic-SR}&\multirow{3}*{只需要访问缓冲区中的数据,带宽较低}&(1)需要复杂的混洗(不经意排序)操作,利用排序网络开销较大\\
        &&(2)平均带宽依然很高,不适用于具体应用\\
        \hline
        \multirow{2}*{IBS-SR}&\multirow{2}*{不需要不经意排序操作,开销降低}&(1)相比Basic-SR增加了客户端空间开销\\
        &&(2)复杂度$O(\sqrt{N})$不够理想\\
        \hline
        \multirow{4}*{层次模型}&\multirow{2}={(1)在平方根模型的基础上优化了带宽}&(1)依然需要复杂的混洗操作\\
        &\multirow{3}={(2)相比于平方根模型,层次模型考虑了每一个数据块被访问的频率}&(2)依据hash函数计算每一个数据块对应的数据块集合,容易导致hash值冲突;\\
        &&(3)平均带宽依然很高,不适用于具体应用\\
        \hline
        \multirow{3}*{分区模型}&\multirow{2}*{(1)分布式ORAM可以适用于云存储场景}&(1)需要更大的客户端存储空间\\
        &\multirow{2}*{(2)在层次模型上进一步优化了带宽}&(2)每一个服务器依然需要复杂的混洗操作,同时,驱逐操作也相对困难\\
        \hline
        \multirow{3}*{树状模型}&\multirow{3}={相比于平方根模型和层次模型,极大地降低了复杂度}&(1)驱逐操作的复杂度很高\\
        &&(2)客户端需要存储一个映射表,增大了客户端的存储空间\\
        \hline
    \end{tabular}
    \caption{几种ORAM模型的优缺点}
    \label{tab:adv}
\end{table}
\subsubsection{性能小结}
前文所述的是一些基本的ORAM架构,表\ref{tab:perf}对它们的性能进行了小结与比较,其中包括计算时间复杂度(摊还、最坏),通讯开销(摊还、最坏),以及服务器与客户端上消耗的空间存储。注:这里假设不经意排序采用$O(N\log^2N)$的Batcher排序网络。
\begin{table}[H]
    \centering
    \begin{tabular}{|c|c|c|c|c|c|c|}
        \hline
        \multirow{2}*{ORAM Type}& \multicolumn{2}{c|}{Computation Overhead}& \multirow{2}*{Cloud Storage}& \multicolumn{2}{c|}{Communication Overhead}& \multirow{2}*{Client Storage}\\
        \cline{2-3}\cline{5-6}
        &Amortized&Worst-Case&&Amortized&Worst-Case&\\
        \hline
        Basic-SR& $O(\sqrt{N}\log^2N)$& $O(N\log^2N)$& $O(N)$& $O(\sqrt{N}\log^2N)$& $O(N\log^2N)$& $O(1)$\\
        \hline
        IBS-SR& $O(\sqrt{N})$& $O(N)$& $O(N)$& $O(1)$& $O(\sqrt{N})$& $O(\sqrt{N})$\\
        \hline
        Basic-HR& $O(\log^4N)$& $O(N\log^3N)$& $O(N\log N)$& $O(\log^4N)$& $O(N\log^3N)$& $O(1)$\\
        \hline
        TP-ORAM& $O(\log N)$& $O(\sqrt N)$& $O(N)$& $O(1)$& $O(1)$& $O(\sqrt{N}+\frac{N}{B})$\\
        \hline
        BB-ORAM& $O(\log^2N)$& $O(\log^2N)$& $O(N\log N)$& $O(\log^2N)$& $O(\log^2N)$& $O(\frac{N}{B})$\\
        \hline
        Path-ORAM& $O(\log N)$& $O(\log N)$& $O(N)$& $O(\frac{\log^2 N}{\log B})$& $O(\frac{\log^2 N}{\log B})$& $O(\log N+\frac{N}{B})$\\
        \hline
    \end{tabular}
    \caption{几种ORAM模型的性能小结}
    \label{tab:perf}
\end{table}

\subsection{ORAM研究进展}
不同的ORAM模型各有优劣,其中,对当下的ORAM研究启发最大的是Hierarchical ORAM (Basic-HR)的层状结构、Path ORAM的树状结构与TP-ORAM的分区思想。ORAM的研究主要集中在减小ORAM的开销上,这是由ORAM本身的特点所决定的。ORAM主要应用于内存或存储系统,如果读写开销与存储开销太大,就会严重限制其实际应用。因此,前文所述ORAM研究主要是通过设计新的数据结构与算法,来降低ORAM的开销。\par
在这些研究的基础上,当下研究注重于结合各种ORAM模型的特点,改进已有的ORAM模型的算法与数据结构,运用新开发的技术,从细节上优化ORAM的带宽与存储开销、减少交互轮数、提高效率等。
\begin{enumerate}
    \item 如\cite{ref11}中提出的\textit{Ring ORAM}模型,对树状模型进行了进一步优化,其借鉴了Partition ORAM (TP-ORAM分区模型)的驱逐操作的思想,设置驱逐频率常数$A$,每$A$步后对路径和缓冲区进行一次驱逐操作,而不用像Path-ORAM那样每次都要讲Stash写回树中,首次使树状模型的复杂度独立于数据块大小$B$,比Path-ORAM效率提升了2.3到4倍\cite{ref11}。
    \item 而ObliviStore\cite{ref10}则是一个高性能、分布式的ORAM存储系统,提出分布式ORAM定义的同时,也实现了异步IO操作。
    \item 还有的模型\cite{ref12}使用有效数据的冗余来代替伪数据块,通过重复读写不同位置上的冗余数据来实现访问模式的混淆,将时间开销缩小到$O(1)$,然而其空间开销巨大(取决于冗余的程度)。
\end{enumerate}

\subsection{存在问题}
然而,虽然当下的ORAM研究已经取得了很大的进展,效率也有所提升,但依然有许多亟待解决的问题,这些问题限制了ORAM的实际应用,导致ORAM系统依然停留在理论研究上。
\begin{enumerate}
    \item ORAM的额外带宽开销依旧很大。如果要将ORAM部署于云环境上,则带宽开销会成为性能的主要瓶颈,严重影响ORAM的实用性。
    \item ORAM需要额外的客户端与服务端空间。ORAM系统在服务端不仅存储真实数据块,还会存储伪数据块,这造成了存储空间的浪费。而ORAM系统在客户端也常常需要存储缓冲区、映射表等,造成客户端空间的开销。
    \item ORAM的访问操作需要与服务端进行多轮交互。与带宽开销相似,这个缺点也会影响其在云环境下的应用性能,同时也会让ORAM的具体实现变得复杂。
    \item 当前支持分布式存储的ORAM极少,仅有少数(如上文提到的ObliviStore\cite{ref10})支持异步访问、并发访问的安全性,但其开销较大,限制了实际应用。
\end{enumerate}

\subsection{研究展望}
如前文所述,ORAM技术最主要的问题就是开销巨大,这导致其难以应用于实际。因此当前的研究着重提高其效率,降低其开销。除此之外,要实际应用ORAM技术还需要考虑更多问题,如分布式系统的同步、异步访问等等。具体来说,未来的ORAM研究可以注重以下几个方面。
\begin{enumerate}
    \item 在减小带宽开销方面还可以进行优化。云环境中,往往存储空间充足而传输带宽受到限制,如果带宽开销很大就会导致传输数据块的时间非常长,限制OARM的使用。当前,有的ORAM(如Onion-ORAM\cite{ref13})能使带宽开销降低到$O(1)$,但计算复杂度高。未来的研究可以尝试寻找其它的方式来达到$O(1)$的带宽开销。
    \item 在空间利用方面可以进行优化。当下的ORAM需要在服务端存储大量的伪数据块,这些伪数据块的内容没有意义,仅用于混淆真实数据块。但是,这样会造成空间的浪费。可以用利用伪数据块来存储一些信息,如\cite{ref12}的冗余思想,使伪数据块的内容有意义,提高空间的利用效率。
    \item 在分布式与并行访问方面进行研究。分布式与并行访问是云环境的一大特点,而真正实用的分布式ORAM系统还有待开发。
    \item 对ORAM模型的进一步完善。ORAM经典模型在上个世纪提出,这是一个简单的模型,只支持读写操作。未来可以针对实用性对ORAM模型进行进一步拓展,完善细节问题,如支持数据的删除与更新操作、满足分布式系统的一致性与可扩展性,实现日志功能等,从而推动ORAM系统的实际应用。
\end{enumerate}